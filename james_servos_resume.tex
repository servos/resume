%!TEX TS-program = xelatex
%!TEX encoding = UTF-8 Unicode
% Awesome CV LaTeX Template for CV/Resume
%
% This template has been downloaded from:
% https://github.com/posquit0/Awesome-CV
%
% Template license:
% CC BY-SA 4.0 (https://creativecommons.org/licenses/by-sa/4.0/)
%


%-------------------------------------------------------------------------------
% CONFIGURATIONS
%-------------------------------------------------------------------------------
% A4 paper size by default, use 'letterpaper' for US letter
\documentclass[11pt, a4paper]{awesome-cv}

\addbibresource{bibliography.bib} % Specify the bibliography file to include publications

% Configure page margins with geometry
\geometry{left=1.4cm, top=.8cm, right=1.4cm, bottom=1.8cm, footskip=.5cm}

% Color for highlights
% Awesome Colors: awesome-emerald, awesome-skyblue, awesome-red, awesome-pink, awesome-orange
%                 awesome-nephritis, awesome-concrete, awesome-darknight
\colorlet{awesome}{awesome-concrete}
% Uncomment if you would like to specify your own color
\definecolor{awesome}{HTML}{666666}

% Colors for text
% Uncomment if you would like to specify your own color
% \definecolor{darktext}{HTML}{414141}
% \definecolor{text}{HTML}{333333}
% \definecolor{graytext}{HTML}{5D5D5D}
% \definecolor{lighttext}{HTML}{999999}
% \definecolor{sectiondivider}{HTML}{5D5D5D}

% Set false if you don't want to highlight section with awesome color
\setbool{acvSectionColorHighlight}{true}

% If you would like to change the social information separator from a pipe (|) to something else
\renewcommand{\acvHeaderSocialSep}{\quad\textbar\quad}


%-------------------------------------------------------------------------------
%	PERSONAL INFORMATION
%	Comment any of the lines below if they are not required
%-------------------------------------------------------------------------------
% Available options: circle|rectangle,edge/noedge,left/right
% \photo[rectangle,edge,right]{./examples/profile}
\name{James}{Servos}
\position{Engineering R\&D Leader{\enskip\cdotp\enskip}Autonomy and Robotics}
\address{Kitchener, Ontario, Canada}

\mobile{+1-519-574-1772}
\email{servos@gmail.com}
%\dateofbirth{January 1st, 1970}
%\homepage{www.posquit0.com}
\github{servos}
\linkedin{james-servos}
% \gitlab{gitlab-id}
% \stackoverflow{SO-id}{SO-name}
% \twitter{@twit}
% \skype{skype-id}
% \reddit{reddit-id}
% \medium{madium-id}
% \kaggle{kaggle-id}
% \hackerrank{hackerrank-id}
\googlescholar{S-cpmfYAAAAJ}{James Servos}
%% \firstname and \lastname will be used
% \googlescholar{googlescholar-id}{}
% \extrainfo{extra information}

%\quote{``Be the change that you want to see in the world."}


%-------------------------------------------------------------------------------
\begin{document}

% Print the header with above personal information
% Give optional argument to change alignment(C: center, L: left, R: right)
\makecvheader[C]

% Print the footer with 3 arguments(<left>, <center>, <right>)
% Leave any of these blank if they are not needed
\makecvfooter
  {\today}
  {James Servos~~~·~~~Resume}
  {\thepage}


\vspace*{-3mm}
%-------------------------------------------------------------------------------
% Summary
%-------------------------------------------------------------------------------
\cvsection{Summary}

\begin{cvparagraph}

Dynamic and accomplished Engineering Leader with over a decade of hands-on experience building robotics products from the ground up, coupled with a profound passion for research and innovation. With a proven track record of delivering cutting-edge robotic systems, I thrive at the intersection of rigorous engineering deveopment and applied academic research. Eager to bring my expertise to a forward-thinking organization committed to pushing the boundaries of technology in the field of robotics.

\end{cvparagraph}

\vspace*{-5mm}
%----------------------------------------------------------------------------------------
%	WORK EXPERIENCE SECTION
%----------------------------------------------------------------------------------------
\cvsection{Work Experience}
\begin{cventries}
\vspace*{-1mm}
\cventry
  {Director, Mobile Robotics Technology}
  {OTTO by Rockwell Automation}
  {Kitchener, Canada}
  {2024--Present}
  {
    \begin{cvitems}
      \item Develop, lead and manage execution of R\&D programs in the areas of navigation, planning, perception, mapping, deep learning, reinforcement learning, and generative AI
      \item Create comprehensive strategic roadmaps for robotics R\&D which have direct impact on advancing mobile robotics and deliver tangible value in real world application.
    \end{cvitems}
  }

\vspace*{-1mm}
\cventry
  {Director, Robotics Software}
  {Locus Robotics}
  {Willmington, USA (Remote)}
  {2023--2024}
  {
    \begin{cvitems}
      \item Responsible for all aspects of robotics software development including direct management, technical oversight and design review.
      \item Develop release plans, priorities, roadmaps, define program objectives and ensure projects stay on track both technically and managerially.
      \item Oversee the six teams that build all aspects of the robot software stack, Localization \& Mapping, Perception, Planning \& Control, Platform, Tools, and Simulation \& Testing.
    \end{cvitems}
  }

\vspace*{-2mm}
\cventry
  {Director, Robot Software Program Management and Core Perception}
  {}
  {}
  {2023--2023}
  {
    \begin{cvitems}
      \item Lead product owner and technical leader respondible for leading, developing, and planing the portfolio of software and technologies which enable autonomous, intelligent, and efficient operation of Locus robots.
      %\item Lead product owner and technical leader of the Robot Software Group
      \item Create and maintaining release plans, development processes, roadmaps, and strategic initiatives to ensure the success of the Robot Software.
    \end{cvitems} 
  }


%------------------------------------------------
\cventry
  {Director, Perception}
  {Clearpath Robotics}
  {Kitchener, Canada}
  {2022--2023}
  {
    \begin{cvitems}
      \item Developing and managing strategic initiatives to push the state-of-the-art in robot perception technologies for industrial indoor robots. 
      \item Create and plan perception projects and technology roadmaps for industrial vehicle autonomy. 
      \item Manage research initiatives with both internal stakeholders as well as external academic and industrial partners. 
      %\item Design and architect perception software across a wide breadth of perception applications.
    \end{cvitems} 
  }

\vspace*{-2mm}
\cventry
  {Autonomy Engineering Mananger - Perception}
  {}
  {}
  {2018--2022}
  {
    \begin{cvitems}
      %\item Managing strategic initiatives in robot perception for industrial indoor robots.
      %\item Develope perception technology road map for vehicle autonomy innovation. 
      %\item Manage research initiatives with internal, academic and industrial partners. 
      %\item Lead a highly skilled group responsible for all robot perception work in the product
      %\item Localization, SLAM, object tracking, object classification, obstacle avoidance, machine learning, semantic understanding and segmentation, among many others.
      %\item Continued responsibility for a growing team of highly skilled perception developers.     
      \item Design and architect perception software across breadth of perception areas.
      \item Scaling the team and technoloy to support rapid growth and adapt to continuously changing state-of-the-art
      \item Lead the introduction, development, and integration of machine learning technology into the robotics stack.
     \end{cvitems} 
  }

\vspace*{-2mm}
\cventry
  {Perception Team Leader} 
  {}
  {}
  {2016 -- 2018}
  {
    \begin{cvitems}
      \item Develop and lead a team of highly talented developers to create state-of-the art algorithms for robotics perception, computer vision, and SLAM.
      \item Develop perception software and features to push the bounds on the reliability, efficiency, and intelligence of our industrial vehicle solutions.
    \end{cvitems} 
  }

\vspace*{-2mm}
\cventry
  {Senior Autonomy Engineer} 
  {}
  {}
  {2015 -- 2016}
  {
    \begin{cvitems}
      \item Design, architect, and develop cutting-edge autonomy software for industrial mobile robotics specializing in perception for mobile robotics including, SLAM, obstacle detection, target tracking, image processing, and long term robust autonomy.
      \item Handled challanging use cases such as highly dynamic unstructured environments and near continuous up time within the industrial setting.
    \end{cvitems} 
  }

\vspace*{-2mm}
\cventry
  {Autonomy Engineer} 
  {}
  {}
  {2014 -- 2015}
  {
  \begin{cvitems}
    \item Development of autonomy software and systems for custom industrial robotics applications and systems.
    \item Architect and developed the core autonomy software that would become the basis for OTTO Motors platforms
  \end{cvitems} 
  }

%------------------------------------------------
\cventry
  {Graduate Student Researcher} 
  {Waterloo Autonomous Vehicles Laboratory - University of Waterloo}
  {Waterloo, Canada}
  {2012--2014}
  {
  %  \begin{cvitems}
  %    \item Research focuses on improving SLAM methods by incorporating multi-channel information from non-homogeneous sensor configurations
  %  \end{cvitems}
  }

%------------------------------------------------
\vspace*{-5mm}
\cventry
  {Embedded Systems Software Developer} 
  {Research In Motion}
  {Waterloo, Canada}
  {2011-2012}
  {
  %\begin{cvitems}
  %    \item Developed sensor drivers and DSP algorithms for mobile phone products. 
  %\end{cvitems}
  }

\end{cventries}

%----------------------------------------------------------------------------------------
%	COMMITTEES
%----------------------------------------------------------------------------------------
\vspace{-8mm}
\cvsection{Comittees}
\begin{cventries}
\cventry
  {Research and Training Committee}
  {Canadian Robotics Council}
  {Canada}
  {2022--Present}
  {
    \begin{cvitems}
      \item Identify and share common challenges and opportunities for Canadian robotics researchers and educators
      \item Facilitate coordinated efforts with industry and the government to stimulate collaborations between larger groups of Canadian roboticists.
    \end{cvitems}
  }

\vspace{-1mm}
\cventry
  {Steering and Scientific Review Committee}
  {NSERC Canadian Robotics Network}
  {Canada}
  {2019--2023}
  {
    \begin{cvitems}
      \item Coordination of research themes, allocation of resources and acceptance of network members for NCRN 
    \end{cvitems}
  }
\end{cventries}

%----------------------------------------------------------------------------------------
%	EDUCATION
%----------------------------------------------------------------------------------------
\cvsection{Education}

%-------------------------------------------------------------------------------
%	CONTENT
%-------------------------------------------------------------------------------
\begin{cventries}

%---------------------------------------------------------
  \cventry
    {Master of Applied Science in Mechatronics Engineering} % Degree
    {University of Waterloo} % Institution
    {Waterloo, Canada} % Location
    {Sept. 2012 - August 2014} % Date(s)
    {
    %  \begin{cvitems} % Description(s) bullet points
    %    \item {Focused on Simultanious Localization and Mapping for mobile robotics}
    %  \end{cvitems
    }

\vspace{-4mm}
  \cventry
    {Batchelor of Applied Science in Mechatronics Engineering} % Degree
    {University of Waterloo} % Institution
    {Waterloo, Canada} % Location
    {Sept. 2007 - April 2012} % Date(s)
    {}

%---------------------------------------------------------
\end{cventries}


%----------------------------------------------------------------------------------------
%	PUBLICATIONS SECTION
%----------------------------------------------------------------------------------------

%\vspace{-3mm}
\cvsection{publications}
  \nocite{*}

\vspace{-3mm}
\printbibsection{patent}{patents} % Print all articles from the bibliography

\vspace{-6mm}
\printbibsection{article}{article in peer-reviewed journal} % Print all articles from the bibliography
\vspace{-6mm}
\printbibsection{inproceedings}{international peer-reviewed conferences/proceedings}


%\input{resume/experience.tex}
%\input{resume/honors.tex}
%\input{resume/certificates.tex}
% \input{resume/presentation.tex}
% \input{resume/writing.tex}
% \input{resume/committees.tex}
%\input{resume/education.tex}
% \input{resume/extracurricular.tex}


%-------------------------------------------------------------------------------
\end{document}
